% !TEX encoding = UTF-8 Unicode
\section{Introduction}

My goal is to develop a framework for reasoning about probability in Coq. I believe this can be useful for a few reasons:
\begin{itemize}
\item To reason about randomized algorithms
\item To reason about systems which operate under uncertain/probabilistic conditions (e.g., a random external environment, or systems which occasionally have random faults)
\item To allow a new sort of probabilistic programming language, where a user first specifies the distribution they wish to understand in a simple way, and then uses a theorem prover to derive efficient (verified) algorithms to sample, compute a PDF, or compute an expectation of the distribution (by composing lemmas of known algorithms/tactics)
\end{itemize}

\subsection{Foundations}

Reasoning about probability in Coq requires developing the theory of probability within Coq. While one could in principle admit the law of the excluded middle as an axiom, and then more-or-less follow measure theory, I don't think this is the right way to go. First of all, most Coq libraries do not admit the law of the excluded middle, and it is potentially possible that some admit axioms which refute the excluded middle. But more importantly, what isn't constructive ends up being useless anyway; if we used measure theory, we would still need to add many computational principles alongside it just to make it useful. By staying constructive, we eliminate the unnecessary distinction between mathematical proofs and computer programs.

Therefore, there is much to gain by developing a theory of probability which uses only constructive principles. The section \emph{Valuations} provides some technical material explaining one possible approach involving formal topologies and continuous valuations.

Errett Bishop claims that the disconnect between probability theory and probability as it is actually used by practitioners is largely explained by the nonconstructivity of classical probability theory\cite{bishop1973}:

\begin{quote}
One suspects that the majority of pure mathematicians... ignore as much content as they possibly can. If this suspicion seems unjust, pause to consider the modern theory of probability. The probability of an event is commonly taken to be a real number between 0 and 1. One might naively expect that the probabilists would concern themselves with the computation of such real numbers. If so, a quick look at any one of a number of modern texts, for instance the excellent book of Doob, should suffice to disabuse him of that expectation. Fragmentation ensues, because much if not most of the theory is useless to someone who is concerned with actually finding probabilities. He will either develop his own semi-independent theories, or else work with ad hoc techniques and rules of thumb.
\end{quote}

In a constructive framework, a probability distribution on $A$ is a function which (as a special case) takes as input a decidable properties on $A$ and returns a constructive real number (that is, an arbitrarily fine rational approximation) between 0 and 1. 

\subsection{Possible directions}

\subsubsection{Synthesis of probability distributions}

Often times, we wish to characterize probability distributions in ways that aren't exactly constructive. For instance, when rolling a die, the key characterization is that landing on any fact is equally likely; or in other words, a permutation of the die faces preserves the measure. From this description, we may want to synthesize a concrete probability distribution which satisfies this property.

The problem of finding probability distributions which satisfy certain properties can be phrased as a Fiat-style deductive synthesis problem. A contradictory characterization (e.g., ``I am certain the die lands on an even number, and that 1 is more likely that 6'') has no synthesis solution, while underspecified characterizations have many synthesis solutions which might have different properties.

Providing a framework for synthesis in this way would help bridge the gap about which Jaynes complains.

\subsubsection{Probabilistic programming language}

Another potential application is to develop a probabilistic programming language embedded in Coq. Probabilistic programming languages allow users to perform probabilistic inference; essentially, the programmer specifies a prior distribution, and then conditions it based on some observation, and wishes to compute some properties of the posterior distribution. There are many ways to learn information about a probability distribution:

\begin{enumerate}
\item sampling from it
\item evaluating a PDF (i.e., Radon-Nikodym derivative) with respect to a better-known distribution
\item computing expectations or moments
\item closed form symbolic definition
\end{enumerate}

For the most part (I believe), probabilistic programming languages only offer the first option. Because their inference algorithms must be general and fully automatic, they are often far less efficient than inference algorithms that could be manually derived.

A probabilistic programming language embedded in Coq could solve many of the shortcomings of existing probabilistic programming languages. First of all, it allows us to separate the problem of specifying a probability distribution from the problem of computing its properties. Users can specify probability distributions by writing probabilistic programs whose meaning is directly interpreted as a probability distribution.

The user can then use either tactics or higher-order programs to learn about the probability distribution they specify. For instance, they may use a rewrite rule to simplify a sum of Gaussians as a single Gaussian.

We can also give a semantics of samplers, particularly in a nice way so that they form a monad. A sampler for a distribution $\mu$ on the space $A$ using a random source distribution $\mu_R$ on the space of random seeds $R$ is a function $ f : R \to R \times A$ such that
\[
  \mathsf{map}\ f \ \mu_R = \mu_R \times \mu,
\]
where the $\times$ operation represents the product of independent distributions.

Deriving \emph{efficient} samplers can then be phrased as a deductive synthesis problem: to sample from a distribution $\mu$, we want to find a random source distribution $\mu_R$ and a sampling function $f$ which samples $\mu$. Because of the compositionality of probabilistic programs as well as samplers, we can derive samplers in a compositional manner. One can imagine a user deriving sampling algorithms using tactic-based process, where a user can use a user-provided database of efficient samplers which have been proven correct, and then compile a sampler by composing these efficient samplers with pre-existing sampling combinators.

We can imagine a situation in the case where a user may want to derive a PDF for their distribution. This also can be phrased as a problem of deductive synthesis.

\subsubsection{Verification of uncertain systems}

We can use theorem provers such as Coq to verify the logical or absolute correctness of software, as well as systems in a more general sense. However, often software and systems operate under uncertain conditions which are best modeled as random. Viewing probability as extended logic, we can naturally generalize the application of theorem provers to verifying that systems provide some guarantee with at least some probability. 

Note that verifying that a system provides a guarantee with at least some probability is a significantly easier task than computing exactly what that probability is.

For instance, one goal of approximate computing is to design systems which operate well even though their components may be occasionally faulty. We expect distributed systems to provide certain guarantees although some nodes of the systems may be faulty. Network communications involve uncertainty as well.

\section{Valuations}

\subsection{References}

\cite{jones1990}: This is basically the first definition of valuations on topologies, and gives a fairly thorough explanation of them and some in-depth, low-level facts about them.

\cite{maietti2005}: First of all, this paper presents ``join-formal topologies", which are what I use to define measure for formal topologies. They require the base to have a join operation in addition to a meet operation, so that the base is actually a lattice.

Second of all, the result of this paper should be useful to me for defining a topology on the space of valuations itself. The domain of a valuation - a formal topology - should have a topology which is locally compact, and the codomain of a valuation should be an inductively generated formal topology, so the result says that the space of valuations can also be given an inductively generated formal topology. This will be key for showing that valuations form a monad over inductively generated formal topologies, I think!

\cite{escardo2004}: This is a really interesting book. It presents topology ``synthetically"; instead of taking topologies as ``a class of subsets such that..." or even as formal topologies, it presents it essentially as a programming language, where the continuity of functions is expressed by the fact that the function can be written in this programming language.

The way that I think that working with formal topologies in Coq will be made pleasant is by embedding such a programming language in Coq, and showing that formal topologies can serve as a model for synthetic topology.

\cite{vickers1989}: I loaned this from the library. It is a rare resource in that it actually tries to motivate the relevance of locales in a comprehensible way. It is, in this way, very friendly.

\cite{coquand2003}. \cite{sambin2000}.

\subsection{Basic valuations}

In this section, we define a notion of a \emph{basic valuation} on a formal topology, which is a function which assigns each basic open weight in a coherent manner. We then show that a basic valuation can be extended to the standard notion of a continuous valuation on the locale determined by the formal topology.

Let $\R$ denote the real numbers (presented as two-sided Dedekind cuts over propositions, not including points at infinity). Let $\lowerT{\R}$ denote the lower real numbers (presented as one-sided lower Dedekind cuts over propositions, including $+\infty$). Similarly, let $\R^+$ denote the non-negative reals (not including $\infty$) and $\lowerT{\R}^+$ the non-negative lower reals (including $\infty$).

Let $(S, \cov, \bullet, \vee, \Pos)$ be a join-closed formal topology, where $S : \Type$, $\cov : S \to (S \to \Prop) \to \Prop$, $\bullet : S \to S \to S$, $\vee : S \to S \to S$ and $\Pos : S \to \Prop$. Let $\mathcal{A} : (S \to \Prop) \to (S \to \Prop)$ be the saturation operator, defined by
\[
\mathcal{A}(U) = \{ a : S \suchthat a \cov U\}.
\]

\subsection{Join closures}
For an open cover $U : S \to \Prop$, we define its join-closure $\bigvee U$ to be the inductive type with constructors
\begin{gather*}
\frac
{a \in U}
{a \in \bigvee U}
\tag{$\bigvee$-refl}
\\
\frac
{a \in \bigvee U \quad b \bigvee \in U}
{a \vee b \in \bigvee U}.
\tag{$\bigvee$-join}
\end{gather*}

\begin{lemma}
\label{bigvee-cov}
If $U \cov V$, then $\bigvee U \cov V$.
\end{lemma}
\begin{proof}
Suppose $a \in \bigvee U$; either it is in $U$, in which case we immediately know $a \cov V$, or it is of the form $a = b \vee c$, where we know $b \cov V$ and $c \cov V$. Then by the $\vee$-left rule, we know $a \cov V$ as well.
\end{proof}

\begin{corrolary}
For all $U : S \to \Prop$, $U =_\mathcal{A} \bigvee U$.
\end{corrolary}

\begin{lemma}
\label{bigvee-subset}
If $U \subseteq V$, then $\bigvee U \subseteq \bigvee V$.
\end{lemma}

\begin{lemma}
\label{bigvee-distr}
For all $U, V : S \to \Prop$, we have $(\bigvee U) \bullet (\bigvee V) \subseteq \bigvee (U \bullet V)$.
\end{lemma}
\begin{proof}
Follows from the distributivity of $\bullet$ over $\vee$ in $S$. Informally,
\[ 
(\bigvee_i u_i) \bullet (\bigvee_j v_j)
= \bigvee_i \bigvee_j (u_i \bullet v_j),
\]
where the left-hand side is in $(\bigvee U) \bullet (\bigvee V)$ and the right-hand side is in $\bigvee (U \bullet V)$.
\end{proof}

\subsection{Basic valuations}

Then a \emph{basic valuation} on $(S, \cov, \bullet, \vee, \Pos)$ is a function $\mu : S \to \lowerT{\R}^+$ such that the following propositions hold:

\begin{gather*}
\frac
{a \cov U}
{\mu(a) \le \mu_\infty(U) }
\tag{monotone}
\\
\mu(a) + \mu(b) = \mu(a \bullet b) + \mu(a \vee b)
\tag{modular}
\end{gather*}
where we define the extension of $\mu$ to open covers $\mu_\infty : (S \to \Prop) \to \lowerT{\R}^+$ by
\[
\mu_\infty(U) = \sup \{ \mu(a) \suchthat a \in \bigvee U \},
\]
and we define the supremum of the empty set as 0.

In the remainder of this section, assume $\mu$ is a basic valuation.

\begin{lemma}
We know that $\mu$ is \emph{strict}, meaning that
\[
\frac
{\Pos(a) \to \mu(a) = 0}
{\mu(a) = 0}.
\tag{strict}
\]
\end{lemma}
\begin{proof}
Suppose we have $a : S$ and know that $\Pos(a) \to \mu(a) = 0$. We know $a \cov \{ a \suchthat \Pos(a) \}$, so by monotonicity $\mu(a) \le \sup_{\{a \suchthat \Pos(a)\}} \mu(a) = 0$.
\end{proof}

\begin{lemma}
If $U \cov V$, then $\mu_\infty(U) \le \mu_\infty(V)$.
\end{lemma}
\begin{proof}
Suppose we lower-bound $\mu_\infty(U)$ with $a \in \bigvee U$. Since $U \cov V$, by lemma \ref{bigvee-cov} we have $\bigvee U \cov V$ and thus $a \cov V$, so by monotonicity of $\mu$ we get $\mu(a) \le \mu_\infty(V)$, thus giving the same lower bound for $\mu_\infty(V)$.
\end{proof}

\begin{corrolary}
If $U \subseteq V$, then $\mu_\infty(U) \le \mu_\infty(V)$. If $U =_\mathcal{A} V$, then $\mu_\infty(U) = \mu_\infty(V)$. In particular, $\mu_\infty(\mathcal{A}U) = \mu_\infty(U)$. Additionally,
\[
\mu_\infty(U) = \sup \{\mu(a) \suchthat a \cov U \}.
\]
\end{corrolary}

\begin{lemma}
We have, for a singleton basic cover, $\mu_\infty(\fun{z}{z = a}) = \mu(a)$.
\end{lemma}

\begin{lemma}
For all $a : S$ and $b : S$, $\mu(a \bullet b) \le \mu(a)$.
\end{lemma}
\begin{proof}
Since $a \bullet b \cov a$, we know $\mu(a \bullet b) \le \mu_\infty(\fun{z}{z = a}) = \mu(a)$.
\end{proof}

\begin{lemma}
$\mu_\infty$ is continuous; that is, for every join-semilattice $L$ and directed set $f : L \to (S \to \Prop)$ (where $i \le j$ implies $f(i) \subseteq f(j)$), we have
\[
\mu_\infty\left(\bigcup_{l : L} f(l)\right) = \sup_{l : L} \mu_\infty(f(l)).
\]
\end{lemma}
\begin{proof}
It is easy that the right-hand side is no greater than the left-hand side, because for each $l : L$, we know $f(l) \subseteq \cup_{l' : L} f(l')$. To see that the left-hand side is no greater than the right-hand side, suppose we lower-bound the left-hand side by finding some $a : S$ such that $a \in \bigvee \cup_{l : L} f(l)$. We claim that there is some $l_a$ such that $a \in \bigvee f(l_a)$, and therefore we can lower-bound the right side. 

We reason by induction on the membership of $a$ in $\bigvee \cup_{l : L} f(l)$. If $a \in \cup_{l : L} f(l)$, then directly we have some $l_a$ such that $a \in f(l_a) \subseteq \bigvee f(l_a)$. In the inductive case, we have $a = b \vee c$, where there are $l_b$ and $l_c$ such that $b \in \bigvee f(l_b)$ and $c \in \bigvee f(l_c)$. Then take $l_a = \max(l_b, l_c)$. Since $\bigvee f(l_b) \subseteq \bigvee f(l_a)$, and likewise for $l_c$, we know that $b, c \in \bigvee f(l_a)$. Therefore $a \in \bigvee f(l_a)$.
\end{proof}


\begin{lemma}
$\mu_\infty$ is modular; that is, for every $U, V : S \to \Prop$, we have
\[
\mu_\infty(U) + \mu_\infty(V) = \mu_\infty(U \bullet V) + \mu_\infty(U \cup V),
\]
where $U \cup V$ is just the union (or disjunction) of predicates.
\end{lemma}
\begin{proof}
First, we see that lower-bounding the left-hand side gives a lower bound to the right-hand side: suppose we have $a, b : S$ where $a \in \bigvee U$ and $b \in \bigvee V$. Since by lemma \ref{bigvee-distr} we have $(\bigvee U) \bullet (\bigvee V) \subseteq \bigvee (U \bullet V)$, we can lower-bound the right-hand side with $a \bullet b \in \bigvee (U \bullet V)$ and $a \vee b \in \bigvee (U \cup V)$; this bound is the same due to the modularity property for $\mu$.

It remains to prove that lower-bounding the right-hand side gives a lower bound to the left-hand side. Suppose we have $x, y : S$ where $x \in \bigvee (U \bullet V)$ and $y \in \bigvee (U \cup V)$. We can find $y_U \in \bigvee U$ and $y_V \in \bigvee V$ such that $y = y_U \vee y_V$.

We claim that there exists $x_U \in \bigvee U$ and $x_V \in \bigvee V$ such that $x \in \bigvee (U \bullet V)$ satisfies $x \le x_U \bullet x_V$. Informally, this follows from the fact that given some $\bigvee_i u_i \bullet v_i \in \bigvee (U \bullet V)$, we have 
\[
\bigvee_i u_i \bullet v_i \le \left( \bigvee_i u_i \right) \bullet \left( \bigvee_j v_j \right),
\]
where $\bigvee_i u_i \in \bigvee U$ and $\bigvee_j v_j \in \bigvee V$.

Then we get
\begin{align*}
\mu(x_U \vee y_U) + \mu(x_V \vee y_V)
  &= \mu((x_U \vee y_U) \bullet (x_V \vee y_V)) + \mu((x_U \vee y_U) \vee (x_V \vee y_V))
  \tag{modular}
  \\ &\ge \mu(x_U \bullet x_V) + \mu(y_U \vee y_V)
  \\ &\ge \mu(x) + \mu(y). \tag{monotone}
\end{align*}
Since $x_U \vee y_U \in \bigvee U$ and $x_V \vee y_V \in \bigvee V$, this gives a lower bound to the left-hand side.
\end{proof}

\begin{theorem}
$\mu_\infty$ is a continuous valuation.
\end{theorem}
 
\begin{proof}
Strictness and monotonicity are obvious, and continuity and modularity follow from lemmas above.
\end{proof}

\subsection{Producing join-closed formal topologies}

It may seem a difficult requirement to generate a join-closed formal topology. In fact, there is a construction to produce a join-closed formal topology given in Lemma 2.13 of Maietti et. al.'s \emph{Predicative exponentiation of locally compact formal topologies over inductively generated ones}. Note that this construction, where the basic opens go from being of type $S$ to $\List{S}$, preserves being a set (inductive generation): if $S$ is a set, then so is $\List{S}$.

Accordingly, one can attempt to see what conditions suffice for a valuation defined on a formal topology that would ensure that the derived valuation for the topology, by using the above-mentioned construction, would be a basic valuation (in particular, the difficulty is in satisfying the modularity condition). We could na\"ively translate the modularity back, but perhaps there is a condition that is easier to prove but still implies modularity.


If we do only start with formal topologies $T$ which are not join-closed, then it seems that our valuation must output real numbers $\R^+$ rather than simply lower reals. Suppose we have a function $\mu : T \to \R^+$ and want to extend it to the join-closure encoded by $\List{T}$. Then, the definition $\mu_\vee : \List{T} \to \R^+$ for the elements generated by the join-closure is uniquely determined according to the recursive definition:

\begin{align*}
\mu_\vee([]) &= 0
\\ \mu_\vee(x :: xs) &= \mu(x) + \mu_\vee(xs) - \mu_\vee(x \bullet xs),
\end{align*}
where $x \bullet xs$ is shorthand for $\map{(\fun{z}{x \bullet z})}{xs}$. The subtraction here shows us that $\mu$ must output a two-sided real number rather than a one-sided one. We note that the definition of $\mu_\vee$ is well-founded since the length of the list will decreases by 1 on each recursive call.

Note that $\mu_\vee$ in fact always outputs real numbers, rather than only lower real numbers. This shows that the construction is limited in its scope; not every basic valuation can be defined in this manner. However, it also demonstrates that the class of basic valuations which output real numbers as opposed to lower real numbers is an interesting one; we will call these \emph{real-valued basic valuations}.

\subsection{Products}

In general, the product of two join-formal topologies is not necessarily a join-formal topology. More concretely, given $S$ and $T$ join-formal topologies, and $s_1, s_2 : S$ and $t_1, t_2 : T$, then there is not necessarily already an element representing $(s_1, t_1) \vee (s_2, t_2)$.

More importantly, this implies that defining the product of two basic valuations is not necessarily straightforward, as it is unclear how to assign a lower-real-valued measure to an open such as $(s_1, t_1) \vee (s_2, t_2)$. However, we can easily define the product of two real-valued basic valuations $\mu_S : S \to \R^+$ and $\mu_T : T \to \R^+$ by first defining $\mu : S \times T \to \R^+$ by
\[
\mu(s, t) = \mu_S(s) \mu_T(t),
\]
and then using the join-closure procedure described in the previous section, so that for example we have
\[
\mu((s_1, t_1) \vee (s_2, t_2)) = \mu(s_1, t_1) + \mu(s_2, t_2) - \mu(s_1 \bullet s_2, t_1 \bullet t_2).
\]

\cite{vickers2011} defines products for lower-real-valued valuations over locales by using integration, i.e.,
\[
\mu_{ST} = a \leftarrow \mu_S \ ; \ \map{(\fun{b}{(a, b))}}{\mu_T},
\]
as well as the mirror image $\mu_{TS}$ and shows that they are equivalent in an analog of Fubini theorem. However, the usually Fubini theorem equates these two definitions to a ``standard", symmetric definition of product measures, whereas this sort of definition is conspicuously absent in \cite{vickers2011}, leading me to think that such a definition is not necessarily straightforward, and possibly explaining the difficulty of doing the same thing for formal topologies.

\section{Probability}

Recall that if $\mu$ is a basic valuation, then $\mu_\infty$ is a valuation. In this section, we will ignore formal topology and just work with locales directly, so let $A$ be a locale and $\mu$ be a valuation. If we add the additional requirement that $\mu(\top) = 1$, then we can call $\mu$ a probability distribution. It turns out that this simple requirement adds a lot of structure. For every probability distribution $\mu : A \to \underline{[0,1]}$ defined on the open sets identified with a locale $A$, there is a probability distribution $\mu^* : A \to \overline{[0,1]}$ defined by
\[
\mu^*(a) = 1 - \mu(a)
\]
on the closed sets identified with the locale $A$, where a closed set $a : A$ here is interpreted as the complement of the open set that it usually represents as a locale.

We can view $\mu$ as a measure on open sets which we continue to give lower bounds, increasing from 0, by finding additional ways to verify the open set, while $\mu^*$ is a measure on closed sets that we can upper bounds, decreasing from 1, by finding additional ways to refute the closed set.

\begin{lemma}
\label{comp-real}
Every complemented member of the frame, that is, every clopen, has a measure which is in fact a real number.
\end{lemma}
\begin{proof}
Let $a : A$ have a complement $\bar{a} : A$. Then by modularity, 
\[
\mu(a) + \mu(\bar{a}) = \mu(a \vee \bar{a}) + \mu(a \wedge \bar{a}) = \mu(\top) + \mu(\bot) = 1 + 0 = 1
\]
Since $\mu(a) + \mu(\bar{a}) = 1$, this means that for every $q : \mathbb{Q}$ such that $q < 1$, we can find $x, y : \mathbb{Q}$ such that $x + y = q$ and $x < \mu(a) < 1 - y$ and $y < \mu(\bar{a}) < 1 - x$. Therefore, if we want to locate $\mu(a)$ to within $\varepsilon$, we simply take $q = 1 - \varepsilon$, giving $x < \mu(a) < x + \varepsilon$.
\end{proof}

\subsection{Conditional probability}

Given $a, b : A$, we say that $\mu(a \suchthat b) : \underline{[0,1]}$ is a conditional probability when
\[
\mu(a \wedge b) = \mu(a \suchthat b) \mu(b).
\]

If $\mu(b) = 0$, then $\mu(a \wedge b) = 0$ as well, and any value will suffice for the conditional probability $\mu(a \suchthat b)$. If we know that $\mu(b) > 0$, then reasoning classically, there is a unique solution for $\mu(a \suchthat b)$ given by
\[
\mu(a \suchthat b) = \frac{\mu(a \wedge b)}{\mu(b)}.
\]

However, is not valid constructively. If we want to give $\mu(a \suchthat b)$ a lower bound, we must give $\mu(a \wedge b)$ a lower bound and $\mu(b)$ an \emph{upper} bound. That is, if we view upper real numbers and lower real numbers as having opposite polarities, then division requires the divisor to have the opposite polarity of the dividend and the result.

Intuitively, if we want to affirm that it is likely to affirm $a$ given that we affirm $b$, we must affirm the worlds where we affirm both $a$ and $b$, and refute the worlds where we refute $b$. Therefore, in order to be able to compute $\mu(a \suchthat b)$ in general, we require that $b$ is clopen --- affirmable and refutable --- so that by lemma \ref{comp-real} we can divide by $\mu(b)$.

\subsection{Simple functions and integration}

Remark: we can add valuations and multiply them by scalars (where the scalars are also lower real numbers).

We define an inductive family $\mathsf{Simple} : \mathbf{Loc} \to \Type$ according to

\begin{gather*}
\frac
{q : \mathbb{Q}^+ \qquad a : A}
{\mathsf{Ind}\ q\ a : \mathsf{Simple}\ A}
\\
\frac
{f : \mathsf{Simple}\ A \qquad g : \mathsf{Simple}\ A }
{\mathsf{Add}\ f\ g : \mathsf{Simple}\ A}.
\end{gather*}

We define the integral of simple functions by
\begin{align*}
   \int_A (\mathsf{Ind}\ q\ a) d\mu &= q \mu(a)
\\ \int_A (\mathsf{Add}\ f\ g) d\mu &= \int_A f d\mu + \int_A g d\mu.
\end{align*}

Each $f : \mathsf{Simple}\ A$ corresponds to a continuous map $\mathsf{eval}\ f: \mathcal{C}(A, \underline{\R}^+)$ in the obvious way.

Given two continuous maps $f, g : \mathcal{C}(A, \underline{\R}^+)$, we say that $f \le g$ if for every $q \in \mathbb{Q}^+$
\[
\{ x \in A \suchthat q < g(x) \} \subseteq \{ x \in A \suchthat q < f(x) \}.
\]
Note that the above condition is written in a very suggestive (misleading) notation, indicating some notion of pointwiseness, that's not really the case. This is a sort of ``horizontal'' definition of when $f \le g$ rather than the more standard ``vertical'' one: we mark a horizontal line in the range, and check that the subset of the domain whose image under $g$ falls below line is included in the subset of the domain whose image under $f$ falls below the line. Normally, we see a ``vertical'' definition, where we consider each point $x$ in the domain, and check that $f(x) \le g(x)$ at that point. But in fact the ``horizontal'' definition allows us to remain pointless.

Then we define the integral of a continuous function as
\[
\int_A f d\mu = \sup \left\{ \int_A g d\mu \suchthat g : \mathsf{Simple}\ A, \mathsf{eval}\ g \le f \right\}.
\]

\subsection{Suprema and fixpoints}
We can take the supremum of a directed set of valuations, and that this in turn allows us to describe fixpoints. This allows definition of the geometric distribution, for instance, as a fixpoint of a coin-flipping process.

\section{Examples}

The simplest space is that with a single point, which we will call $\One$. It has no non-trivial open sets. We define its formal topology with the inductive types $S_\One, \cov_\One, \ltimes_\One$ generated by
\begin{gather*}
\top : S_\One
\\ \top \cov_\One \top
\\ \top \ltimes_\One \top
\end{gather*}

It should come as no surprise that there is only a single probability distribution on $\One$, which puts all its mass on the only point that there is:
\[
\mu_\One(U) = \chi(\top \in U).
\]

Perhaps the next simplest space the Sierpínski space $\Sigma$. This space has two points, $\top_\Sigma$ and $\bot_\Sigma$, where the singleton set $\{\top_\Sigma\}$ is open while the singleton set $\{ \bot_\Sigma \}$ is closed. These sets are the only non-trivial open and closed sets, respectively. Since we're working with formal topology, however, we don't ever need to specify the points directly. We use $S_\Sigma, \cov_\Sigma, \ltimes_\Sigma$ generated by
\begin{gather*}
S_\Sigma = 0_\Sigma \quad | \quad 1_\Sigma
\\ 0_\Sigma \cov_\Sigma 1_\Sigma
\\ 1_\Sigma \ltimes_\Sigma 1_\Sigma
\end{gather*}

Here, we should have a bijection between lower reals $r : \underline{[0,1]}$ and probability distributions on $\Sigma$, given by the map $\mu_\Sigma : \underline{[0,1]} \to S_\Sigma \to \underline{[0,1]}$ defined as
\begin{align*}
    \mu(r)(0_\Sigma) &= r
\\  \mu(r)(1_\Sigma) &= 1.
\end{align*}

This means that we observe the open point $\top_\Sigma$ with probability $r$ and refute the closed point $\bot_\Sigma$ with probability $1 - r$.

Given formal topologies $S$ and $T$, a continuous relation $F : S \to T \to \Prop$ and a probability distribution $\mu : A \to \underline{[0,1]}$, we can produce a new probability distribution $\map{f}{\mu} : B \to \underline{[0,1]}$ given by
\[
\map{f}{\mu}(b) = \mu_\infty(\{ s : S \suchthat s\ F\ b \}).
\]

Recall that we can define a point $x$ in the formal topology $S$ as a continuous relation $x : \One \to S \to \Prop$. Then the Dirac delta of this distribution, $\delta_x : S \to \underline{[0,1]}$ is simply given by $\map{x}{\mu_\One}$.

\section{Computation with formal topology}

The collection $\mathsf{Pt}(S, \le, \cov)$ of points on a space $(S, \le, \cov)$ are predicates $ x : S \to \Prop$ satisfying the following rules:
\begin{mathpar}
\inferrule* [right=inhabited]
  { }
  {\exists a, x \models a}
  
\inferrule* [right=convergent]
  {x \models a \\ x \models b}
  {x \models a \downarrow b}

\inferrule* [right=split-cov]
  {x \models a \\ a \cov U}
  {\exists b \in U, x \models b}.
\end{mathpar}
Each of these three rules can be read computationally. We view a point as an interaction structure which provides an interactive refinement process. First, by \irule{inhabited}, we are given an open which the point lies in; this can be viewed as the ``starting state'' for the refinement process. Then, using the \irule{split-cov} rule, we can use an open cover to refine our state of knowledge about the point to one of the opens in the cover. Using the convergent rule, we can combine any of the previous opens in which the point is known to lie into a single open which is as least as fine as all the others, so that we need only keep track of a single ``smallest'' open which a point lies in.

Are the computational rules sufficient for characterizing points? That is, if two points $x$ and $y$ differ extensionally (i.e., $\neg (\forall a : S, a \in x \leftrightarrow a \in y)$), can we determine this by computing with the above rules alone? Suppose the space $S$ is such that there is an open set $\top : S$ such that $\forall a : S, a \le \top$. Furthermore, suppose that there is a point $\bot \subseteq S$ characterized by $\bot = \{ \top \}$. Then, since every point splits covers, the fact that $\bot$ is a point means that every open cover of $\top$ must already \emph{contain} $\top$ since $\bot$ splits covers. Note that $S$ is compact and is not Hausdorff in this scenario.

Moreover, any point in this space may legitimately behave computationally as if it were the point $\bot$. Of course, a point which is not $\bot$ may also behave computationally in a way which clearly distinguishes it from $\bot$. So different ``proofs'' that a point obeys the required rules result in different computational behavior.

There is a way to characterize the potential computational behaviors of points by looking at the specialization preorder on points. Given two points $x \subseteq S$ and $y \subseteq S$, we say $x \sqsubseteq y$ if $x \subseteq y$, that is, every open which contains $x$ also contains $y$. Accordingly, $\bot$ is the bottom element of this order.

It is my impression that two distinct minimal points $x$ and $y$ should be able to be distinguished, but I can't seem to think of a proof. In particular, in a Hausdorff space, the specialization preorder is the discrete preorder (i.e., it is the reflexive relation): all points are incomparable, so all points are minimal, so every point should be able to be distinguished from one another.

So even though we consider points to be determined by the opens in which they lie, different proofs of being a point have different computational behavior, and some may be more useful.

\begin{definition} A \emph{point implementation} of a given point is the collection of proofs that $x$ satisfies the rules \irule{inhabited}, \irule{convergent}, and \irule{split-cov}.
\end{definition}

\begin{definition}
A point implementation of a point $x$ is \emph{maximally informative} if, for every $a : S$ such that $x \models a$, we can derive $\exists b, b \in \downarrow \{ a \} \wedge x \models b$ solely from repeated applications of the \irule{inhabited}, \irule{convergent}, and \irule{split-cov} rules.
\end{definition}

Therefore, maximally informative point implementations allow a point to be uniquely identified.
\begin{theorem}
For the Sierpíski space $\Sigma$, there is a maximally informative point implementation for the indicator $\chi_P$ of a proposition $P : \Prop$ if and only if $P$ is decidable.
\end{theorem}
\begin{proof}
Let $\bot, \top : \Open{\Sigma}$ with $\bot \le \top$ be the basic opens of $\Sigma$. 

$\Longleftarrow$: If $P$ is decidable, then if $P$ is true we can use $x \models \bot$ for the \irule{inhabited} rule, and if $P$ is false, then for all rules we will have $x \models \top$.

$\Longrightarrow$: If we have a maximally informative point implementation for the indicator $\chi_P$ by $\irule{inhabited}$ we have either $x \models \top$ or $x \models \bot$. If $x \models \bot$, then $P$ is true and we are done. If $x \models \top$, then we apply the \irule{split-cov} rule with the cover $\top \cov \{ \bot, \top \}$. If the response is $x \models \bot$, then $P$ is true, and otherwise, $P$ must be false, though the formal proof that $P$ is probably not very simple.
\end{proof}

There is a related notion that I have thought of outside the context of formal topology. Let $\mathsf{Partial} : \Type \to \Type$ denote the ``non-termination'' monad. 
\begin{definition}
For a given proposition $P : \Prop$, a \emph{verified semidecision procedure} for $P$ is a function $d_P : \mathsf{Partial}\ P$ such that
\[
P \to (\exists p : P, d_P \Downarrow \mathsf{Now}\ p).
\]
\end{definition}

\begin{claim}
If $P$ and $Q$ each have verified semidecision procedures, then so do $P \vee Q$ as well as $P \wedge Q$.
\end{claim}

I don't think it matters if the proof $P$ on the left-hand side of the above implication is truncated or not. Similarly, I believe that if $P$ has a verified semidecision procedure, then $P$ has a weakly constant endomap (see \cite{kraus2014}). For any $f : \nat \to \bool$, it should be possible to create a verified semidecision procedure for the proposition $\exists n : \nat, f(n) = \mathsf{true}$.

Viewed in this light, it seems that the simplest encoding of the Sierpínski space $\Sigma$ may not be the best space for creating maximally informative points; I would like that maximally informative points on the Sierpínski space exactly correspond with verified semidecision procedures, but with these definitions this is not the case. Consider the canonical proposition $\exists n : \nat, f(n) = \mathsf{true}$. With this encoding of the Sierpínski space, we cannot encode with the opens how much of the sequence we have already investigated, and so it is not possible to investigate arbitrarily far.

However, I think we can achieve this with a different encoding of the Sierpínski space (i.e., a space which is homeomorphic to Sierpínski but not intensionally equal). For instance, we could take the space $\nat^\infty$, the Alexandroff compactification of $\nat$, where $\infty : \nat^\infty$ indicates non-termination and $n : \nat^\infty$ for finite $n$ indicates termination after $n$ steps. Then, we can take the following quotient: we simply identify all basic opens which $\infty$ does not lie in, and identify all basic opens which $\infty$ does lie in. This should give the Sierpínski space (or, a space homeomorphic to the Sierpínski space), but where there are still the same basic opens and the same covering axioms (though there are new covering axioms as well) as $\nat^\infty$.

\begin{claim}
By defining the Sierpínski space as a quotient of $\nat^\infty$, if a proposition $P : \Prop$ has a (verified) semidecision procedure, then there is a maximally informative point implementation for its indicator $\chi_P$ in the Sierpínski space.
\end{claim}


\subsection{Sampling from probability distributions}

[[ The following is wrong; it doesn't necessarily produce the right distribution. I thought I knew how to sample from a distribution, but this doesn't work. I hope there's a way to fix the argument. ]]

It should be possible to sample from probability distributions in the following sense. Suppose we have a probability distribution $\mu : S \to \lowerT{\R}^+$. Then $\mu_\infty(S) = 1$, so for an open cover $S \cov U$, $\mu_\infty(U) = 1$. Therefore, for every (small) $\varepsilon > 0$, there is a finite subset $\{u_1, \ldots, u_n \}$ such that
\[
\mu(u_1 \vee \cdots \vee u_n) \ge 1 - \varepsilon.
\]

Then, if we define
\[
q_i = \mu(u_1 \vee \cdots \vee u_i),
\]
we have that we can ``fill up'' the interval $[0, 1 - \varepsilon]$ with the rationals $q_i$.

Therefore, we can sample an $x : \R$ uniformly from the interval $[0, 1]$. With probability 1, we can find some $\varepsilon > 0$ such that $x < 1 - \varepsilon$, and then with probability 1 we will find that there is some $i$ such that $q_{i-1} < x < q_i$. In this case, we will say that our randomly sampled point lies within the open $u_i$.

\section{Future plans}

Here I outline what still must be done, both in terms of what is not yet formalized in Coq, and what is not yet clear to me even outside of Coq:

\begin{enumerate}
\item Get suitable definitions of the lower real numbers and non-negative lower real numbers in the framework of formal topology:
\begin{enumerate}
  \item I'm still not sure what the best way is to do define the non-negative lower reals. Perhaps just define the space as the lower reals, where the function $\max(0, \cdot)$ is pre-composed onto all functions and points.
  \item Define simple functions and integration.
  \item Show that every continuous function from $A \to \underline{\mathbb{R}}^+$ is equivalent to the supremum of simple functions.
  \item Show that every continuous function from $A \to \underline{\mathbb{R}}^+$ is equivalent to the supremum of simple functions. This is done in \cite{jones1990}, and the argument there should work here as well.
\end{enumerate}

\item Show that the functor $\mathsf{Prob} : \mathbf{IgTop} \to \mathbf{IgTop}$ is monadic, where $\mathbf{IgTop}$ is the category of inductively generated formal topologies. This should be possible because a probability distribution on a space $A$ is a function from the opens of $A$ to $\underline{\mathbb{R}}^+$, and the topology on the opens of $A$ should be locally compact (since we give it the Scott topology/specialization topology), and then we follow the argument in \cite{maietti2005}.

This will allow people to describe probability distributions in the familiar monadic style.

\item Show that Fubini theorem holds. It is proved in \cite{vickers2011}. This is important for reasoning about probability distributions: If I draw randomly from $\mu : \mathsf{Prob}\ A$ and then randomly from $\nu : \mathsf{Prob}\ B$ and put them together, it's the same as if I did it the other way around.

\item Think about connecting the world of Coq to the world of topological spaces. Since there are topological models for (at least) simply-typed lambda calculus, we should be able to have an Ltac that converts (at least some simple) Coq functions to their denotations as continuous maps in a topological model. 

\item Having to work with join-closed formal topologies is quite onerous, because the join-closed formal topologies are much larger than standard formal topologies. Particular, one might wish that they could prove the modularity condition for elements of a formal topology, and have that imply the modularity of the join-closure of that formal topology. It intuitively makes sense that this might be the case.

\item Connect the definition of local metric completion for formal topology which I have defined to the definitions of metric completions and metric spaces in CoRN\cite{oconnor2008}. This will allow us to define the familiar distributions on real numbers, such as the normal distribution, using the results already available in CoRN. 
\end{enumerate}

\section{Comparison to computable measure theory}

There are interesting connections between computable measures.

\subsection{References}

\cite{schroder2007} describes the equivalence between measures described by samplers and measures which are functions from subsets to real numbers.

In \cite{weihrauch2014}, definition 4.2 describes what, if I understand correctly, looks like it corresponds to maps names for the generating ring (like the base for formal topology) to lower, upper, and plain-old real numbers. This paper is really difficult to read. Though, in the conclusion they say, "From all the representations of measurable sets mentioned in this article only for the representation ?+ intersection and countable union are computable. Therefore, we claim that ?+ is the most useful one for studying computability of measurable functions." That seems to be related to the idea that valuations on formal topologies should map to lower real numbers.

I found \cite{collins2014} to be one of the more understandable papers on computable measure theory. It relates things closely to valuations.

\section{Verified samplers}

A library for topology and probability in Coq allows for the description of probabilistic programs and for reasoning about them. By using formal topology, there will be a strong notion of \emph{executing} continuous functions. However, what notions of execution do we have for probabilistic programs?

One such notion is the evaluation of probabilistic programs on subsets. Given a measure $\mu : \Dist{A}$ and an open set $U : \Open{A}$, we get a lower real number indicating the likelihood of that open set. If $\mu$ is a probability distribution and $U$ is clopen, then we get a bona fide real number as a result.

But another notion of execution of a probabilistic program is sampling. It seems that \cite{schroder2007} indicates that it is possible to convert a probability distribution described in my representation to a sampler, but it is necessarily a kind of metatheoretical result (in the sense that the type $A$ is required to be a quotient of a countably based (QCB) space), and I'm not entirely sure of its applicability, and I doubt the resulting samplers would be efficient.

In practice, probabilistic programs may use random generators that pretend to represent a certain source of randomness, or they may in fact use a physical source of randomness that provides a random stream of some sort. Then \emph{deterministic} programs process this randomness to transform it into the desired probability distribution, and sampling is achieved in this manner.

These are the sorts of programs that are \emph{actually} written and can actually be sampled, so it makes sense to give a semantics to these sampling-based programs and provide tools for synthesizing and verifying programs of this sort.

We can model the source of randomness a probability distribution $\mu_R : \Dist{R}$ over a space $R$, so that these programs are functions $f : R \leadsto A$ (for some space $A$) which sample from the distribution $\map{f}{\mu_R} : \Dist{A}$.

However, this class of functions is not convenient in terms of its compositional properties. For instance, it is possible to have functions $f : R \leadsto A$ and $g : R \leadsto B$ which sample the distributions $\mu_f = \map{f}{\mu_R}$ and $\mu_g = \map{g}{\mu_R}$, and yet there may be no $h : R \leadsto A \times B$ such that $h$ samples $\mu_f \times \mu_g$ (for instance, consider $\mu_R = \mu_f = \mu_g = \coinflip$; we can't generate two coinflips from one).

We can define a preorder on the set of probability distributions by defining $\mu \le \nu$ if there is a function $f$ such that $\mu = \map{f}{\nu}$. The entropy $H(\cdot)$ of a probability distribution is then monotone with respect to this preorder. That is,
\begin{theorem}[Data processing inequality (DPI)]
If $\mu = \map{f}{\nu}$, then $H(\mu) \le H(\nu)$.
\end{theorem}
This suggests that if we want to be able to write samplers for distributions with arbitrarily large (or infinite) entropy, we require that $\mu_R$ has infinite entropy.

The following definition of samplers has better compositionality properties.
\begin{definition}
A \emph{sampler} for a distribution $\mu$ on the space $A$ using a random source distribution $\mu_R$ on the space of random seeds $R$ is a function $ f : R \leadsto R \times A$ such that
\[
  \mathsf{map}\ f \ \mu_R = \mu_R \times \mu.
\]
\end{definition}

Samplers form a monad, in the sense that we can make the definitions
\begin{align*}
\mathsf{unit} &: A \times R \leadsto R \times A
\\ \mathsf{unit}(x, r) &= (r, x)
\\ \mathsf{bind} &: R \leadsto R \times A \to A \times R \leadsto R \times B
  \to R \leadsto R \times B
\\ \mathsf{bind}(f)(g)(r) &= \mathsf{let}\ (r', x) = f(r)\ \mathsf{in}\ g(x, r')
\end{align*}
where $\mathsf{unit}$ samples the Dirac distribution $\delta : A \leadsto \Dist{A}$,
and if $f$ samples $\mu_f : \Dist{A}$ and $g$ samples $\mu_g : A \leadsto \Dist{B}$, then $\mathsf{bind}(f)(g)$ samples $\mathsf{bind}(\mu_f)(\mu_g)$. This means that we can sample deterministic values, and we can compose samplers together. In a sense, this is the familiar ``state'' monad with some extra properties. In particular, if we can sample $\mu$ and also sample $\nu$, we can sample $\mu \times \nu$.

Recalling that the entropy of a product distribution is the sum of the entropies, i.e.,
\[
H(\mu \times \nu) = H(\mu) + H(\nu),
\]
we can conclude that if we have a sampler for some distribution $\mu$ using the random source $\mu_R$, then using the DPI we discover that
\[
H(\mu_R) + H(\mu) \le H(\mu_R),
\]
meaning that if $\mu$ has non-zero entropy (i.e., any randomness at all), then $\mu_R$ must have infinite entropy.

Notice that if we can sample $\nu$ from $\mu_R$ and if $\mu = \map{f}{\nu}$ for some $f$, then we can also sample $\mu$ from $\mu_R$. This is a useful principle for constructing new samplers. For instance, we have these constructions using this principle:
\begin{enumerate}
\item If we can sample $\prod_\omega \coinflip$ (the distribution of infinite sequences of unbiased coinflips), then we c sample uniformly from the real numbers over the unit interval $\mathcal{U}[0,1]$ by exhibiting a continuous map $f : \bool^\omega \leadsto \R$ such that $\map{f}{\prod_\omega \coinflip} = \mathcal{U}[0,1]$. The map $f$ would compute a real number from its binary expansion.
\item Using the Box-Muller transform, if we can sample $\mathcal{U}[0,1]$, we can sample from a Gaussian distribution. I expect this to be a nightmare to prove.
\end{enumerate}

One particularly nice class random sources to consider is, for any $\mu_R : \Dist{R}$, the distribution $\prod_\omega \mu_R : \Dist{A^\omega}$, the distribution of infinite i.i.d. sequences drawn from $\mu_R$ (and within this, it is particularly nice to consider $\mu_R = \coinflip$). There are some interesting samplers that we can create with this source:
\begin{enumerate}
\item Sample a single $\mu_R$, or sample the product distribution $\prod_{k=1}^n \mu_R$ for any $n : \nat$.
\item Sample $\prod_\omega \mu_R$ by splitting the stream of randomness into two.
\end{enumerate}

\subsection{Digression: optimally efficient samplers}

As the saying goes, ``waste not, want not.'' Not all samplers are as good as others. For instance consider the random seed distribution $\prod_\omega \coinflip$ on the space $\bool^\omega$ and suppose we want to sample a random coinflip. Here are two possible implementations:

\begin{align*}
\mathsf{flip}_1 &: \bool^\omega \leadsto \bool^\omega \times \bool
\\ \mathsf{flip}_1 (x \cons xs) &= (xs, x)
\\ \mathsf{flip}_2 &: \bool^\omega \leadsto \bool^\omega \times \bool
\\ \mathsf{flip}_2 (x \cons \_ \cons \_ \cons \_ \cons \_ \cons xs) &= (xs, x)
\end{align*}

We prefer $\mathsf{flip}_1$ over $\mathsf{flip}_2$, because $\mathsf{flip}_2$ wastes precious randomness, whereas $\mathsf{flip}_1$ uses only what it needs. We can quantify this notion of efficiency, and say that $\mathsf{flip}_1$ is optimally efficient in a precise sense.

We will need to introduce some new language. If we have a measure $\mu : \Dist{A}$ and a function $f : A \leadsto A$ such that $\map{f}{\mu} = \mu$, we say that $f$ is a measure-preserving transformation. If $\alpha$ is a finite set of sets in the $\sigma$-algebra generated by the $A$'s topology which is pairwise disjoint and whose union is the whole space, we say that $\alpha$ is a finite partition of $A$. For two finite partitions $\alpha$ and $\beta$, we can define their common refinement $\alpha \vee \beta$ as the intersections of subsets from $\alpha$ and $\beta$. We can define the entropy of a partition $\alpha = \{\alpha_1, \ldots, \alpha_n\}$ by
\[
H(\mu, \alpha) = \sum_{i = 1}^n \mu(\alpha_i) \lg \frac{1}{\mu(\alpha_i)}.
\]
Then we define
\[
h_\mu(f, \alpha) = \lim_{n \to \infty} \frac{1}{n} H\left(\mu, \bigvee_{k = 1}^n f^{-k}\alpha  \right).
\]
With this in place, we can define the Kolmogorov-Sinai entropy of a measure-preserving map $f$ of $\mu$:

\begin{definition}
The \emph{Kolmogorov-Sinai entropy} of a measure-preserving map $f$ of $\mu$ is
\[
H_\mu(f) = \sup \{ h_\mu(f, \alpha) \suchthat \text{$\alpha$ is a finite partition of $A$} \}.
\]
\end{definition}

I'd like to formulate an equivalent definition. Note every finite partition $\alpha$ is in bijective correspondence with a measurable function $g_\alpha : A \leadsto F_n$ for some finite space $F_n = \{1, \ldots, n \}$ such that $g_\alpha^{-1}(k) = \alpha_k$, meaning that $H(\mu, \alpha) = H(\map{g_\alpha}{\mu})$. We then have
\[
h_\mu(f, \alpha) = h_\mu(f, g_\alpha) = \lim_{n \to \infty} \frac{1}{n} H\left( \map{\left(\prod_{k = 1}^n g_\alpha \circ f^k\right)} \mu \right),
\]
where the product of measurable maps $f \times g$ is defined by $(f \times g)(x) = (f(x), g(x))$.

This gives an alternative definition of the Kolmogorov Sinai entropy which uses measurable functions instead of partitions, and entropy of probability distributions rather than entropy of partitions:
\[
H_\mu(f) = \sup \left\{ h_\mu(f, g) \suchthat \text{$F$ is a finite space, $g : A \leadsto F$ is measurable} \right\}.
\]

Now, suppose that we have some $g$ such that $\map{g}{\mu}$ is independent from $\map{f}{\mu}$. By induction, it is clear that for all $k \ne \ell$, $\map{(g \circ f^k)}{\mu}$ is independent from $\map{(g \circ f^\ell)}{\mu}$, and therefore,
\begin{align*}
h_\mu(f, g) 
&= \lim_{n \to \infty} \frac{1}{n} H\left( \map{\left(\prod_{k = 1}^n g \circ f^k\right)} \mu \right)
\\ &= \lim_{n \to \infty} \frac{1}{n} H\left( \prod_{k = 1}^n \map{(g \circ f^k)}{\mu} \right)
\\ &= \lim_{n \to \infty} \frac{1}{n} \sum_{k = 1}^n H\left( \map{(g \circ f^k)}{\mu} \right)
\\ &= \lim_{n \to \infty} \frac{1}{n} \sum_{k = 1}^n H\left( \map{g}{\mu} \right)
\\ &= H(\map{g}{\mu}).
\end{align*}

Note that whenever we have a sampler $f : R \leadsto R \times A$ of a distribution $\mu : \Dist{A}$ using randomness $\mu_R : \Dist{R}$, we have that $\pi_1 \circ f$ is a measure-preserving transformation of $\mu_R$, and $\map{(\pi_2 \circ f)}{\mu}$ is independent from $\map{(\pi_1 \circ f)}{\mu}$, so if $A$ is finite, we are in precisely this situation.
\begin{theorem}
If $f : R \leadsto R \times A$ is sampler of a distribution $\mu : \Dist{A}$ using randomness $\mu_R : \Dist{R}$, then
\[
H_{\mu_R}(\pi_1 \circ f) \ge H(\mu).
\]
\end{theorem}
\begin{proof}
If $A$ is finite, then 
\[
h_{\mu_R}(\pi_1 \circ f, \pi_2 \circ f) = H(\map{(\pi_2 \circ f)}{\mu_R}) = H(\mu),
\]
so
\[
H_{\mu_R}(\pi_1 \circ f) \ge h_{\mu_R}(\pi_1 \circ f, \pi_2 \circ f) = H(\mu).
\]
The proof can be extended in a straightforward way if $A$ is not finite (using the fact that entropy can be continuously approximated by finite partitions).
\end{proof}

This theorem says that if we want to produce a distribution with a certain amount of randomness, the function we apply to the random source must be at least that random/chaotic. It suggests a definition of optimally efficient samplers:

\begin{definition}
A sampler $f : R \leadsto R \times A$ of a distribution $\mu : \Dist{A}$ using randomness $\mu_R : \Dist{R}$ is \emph{optimally efficient} if
\[
H_{\mu_R}(\pi_1 \circ f) = H(\mu).
\]
\end{definition}

Returning to the original example, we have that $H(\coinflip) = 1\un{bit}$, $H_{\prod_\omega \coinflip}(\pi_1 \circ \mathsf{flip_1}) = 1 \un{bit}$ and $H_{\prod_\omega \coinflip}(\pi_1 \circ \mathsf{flip_2}) = 5 \un{bits}$, so $\mathsf{flip_1}$ is an optimally efficient sampler, while $\mathsf{flip_2}$ is not.

Kolmogorov-Sinai entropy is a fairly sophisticated concept, so I'm not sure that we will be able to do much in the way of formal proof regarding these definitions. However, I think this notion of sampler efficiency enriches the theory of random samplers.

\subsection{Partial samplers}

Unfortunately, many interesting samplers are not total. For instance, consider the sampler of a biased coinflip with a bias with the binary expansion $0.b_1 b_2 \ldots$ from a stream of unbiased coinflips. We would like to define the sampler as follows:
\begin{align*}
\mathsf{bias} &: \bool^\omega \times \bool^\omega \leadsto \bool^\omega \times \bool
\\ \mathsf{bias}(b \cons bs, r \cons rs) &= \mathsf{if}\ b < r\ \mathsf{then}\ (rs, \mathsf{false})
  \ \mathsf{else}\ \mathsf{if}\ b > r\ \mathsf{then}\ (rs, \mathsf{true})\ \mathsf{else}\ \mathsf{bias}(bs, rs)
\end{align*}

But in the case that $b \cons bs = r \cons rs$, this function does not terminate! The recursion in the above definition is not well-founded. Fortunately, this event occurs with measure 0, because the randomness ($\prod_\omega \coinflip$) is distributed uniformly at random.

The solution is to use the topological construction of \emph{lifting} to allow general fixpoints. Given any space $A$, we can adjoin a bottom element $\bot_A$ and modify the topology, so that the lifted (as well as compactified) space $A_\bot$ has least fixed points. We think of $\bot_A$ as indicating nontermination. This construction is just the generalization of lifted domains in domain theory. So the correct type of $\mathsf{bias}$ is
\[
\mathsf{bias} : \bool^\omega \times \bool^\omega \leadsto \left( \bool^\omega \times \bool \right)_\bot.
\]

Since $\Open{A} \subseteq \Open{A_\bot}$\footnote{Maybe it is more correct to say $\Open{A} \hookrightarrow \Open{A_\bot}$.}, it is straightforward to define $\mathsf{unlift} : \Dist{A_\bot} \to \Dist{A}$. For a partial measure $\mu : \Dist{A_\bot}$, we will have
\[
\mu(A_\bot) = \mathsf{unlift}(\mu)(A)
\]
if $\mu$ ``terminates'' with probability 1. Now we can define a relaxed notion of samplers which are allowed to be partial:

\begin{definition}
A \emph{partial sampler} for a distribution $\mu$ on the space $A$ using a random source distribution $\mu_R$ on the space of random seeds $R$ is a function $ f : R \leadsto \left(R \times A \right)_\bot$ such that
\[
 \mathsf{unlift}\left(\mathsf{map}\ f \ \mu_R\right) = \mu_R \times \mu.
\]
\end{definition}

The definition of a partial sampler necessarily implies that $f$ must terminate with probability 1. Note that every total sampler can be turned into a partial sampler, and that partial samplers enjoy the same compositionality properties as total samplers.

With partial samplers, we can construct many more interesting classes of samplers:

\begin{enumerate}
\item Sample the geometric distribution from $\prod_\omega \coinflip$ by drawing coinflips until we get a ``heads'' value.
\item We can sample a biased coinflip from a stream of unbiased coinflips using the example construction discussed above.
\item If we can sample a real number from $\mathcal{U}[0,1]$, then we can sample a biased coinflip (Bernoulli distribution) with weight $p \in [0,1]$. This is achieved by doing an order comparison on the real number that we draw with the bias $p$. This doesn't terminate, but it does terminate with probability 1! Note that this construction is very wasteful in its randomness compared to $\mathsf{bias}$.
\item We can generate a Poisson distribution according to the algorithm in \url{https://en.wikipedia.org/wiki/Poisson_distribution#Generating_Poisson-distributed_random_variables}. Note that a straightforward translation of this algorithm would be wasteful in its usage of randomness, generating entire real numbers, while in reality we only need a finite amount of randomness.
\item We can formulate the general principle of rejection sampling.
\item We can formulate the general principle of inverse transform sampling. Again, it would be interesting to see how we can design an inverse transform sampling so that it is not wasteful in its use of randomness.
\end{enumerate}

\section{Serendipity vs. semi-decision}

The slogan is \emph{serendipity} rather than semi-deciders. Steve Vickers used the term \emph{serendipity} to indicate this distinction, saying one of the concepts characterizing ``observability'' is [[cite Geometric theories and databases]]
\begin{quote}
\emph{serendipity}: one is told how to know in retrospect when one has observed something, but not any method that's guaranteed to result in the observation whenever possible.
\end{quote}

In another example, Vickers gives as an example of an observable property the existence of the Loch Ness monster, saying that
\begin{quote}
we can well imagine under what conditions we could know that the assertion is true. To be sure, this would take a lot of luck, serendipity or Divine Grace, but no particular effort -- in fact we can imagine others putting much time and resources into a vain attempt to repeat our observation.
\end{quote}

The connection with proofs should be obvious. An observable property is one for which we can check (i.e., decide) whether a putative proof is in fact a valid proof. So any type or proposition in Coq corresponds informally to an observable property in this sense. Now, given any proposition for which proof checking is decidable, we can semi-decide whether the proposition in fact holds, by checking all possible proofs. This is the connection between serendipity and semi-decision. For instance, consider the ``or'' operation. In the ``serendipity'' view, we serendipitously hope that we fall into either of the two explicit cases of the overlapping pattern match (rather than the catch-all case, which of course we fall in to), and if we are so lucky to do so, then our output will also luckily fall into the smaller open of the Sierp\'i­nski space. In effect, what the $\mathsf{cases}$ definition tells us is when it suffices to have proved the disjunction. The semi-decision view, instead of considering criteria for satisfactory proof, works with semi-decision procedures for finding proofs themselves, and therefore, to find a proof of the disjunction, the two proof finders must be interleaved.

I happen to favor the \emph{serendipity} view, because one can always go from serendipity to semi-decision, but not the other way around, and also because (at least naive) use of the semi-decision approach could result in some very slow programs, whereas in the serendipity point of view, we are required to provide proofs in order for computation to proceed. This is more work (you might complain about writing proofs), but it means that the resulting code will be more efficient. [Explaining this with an example would be a good idea.]

This distinction is a key difference between the formal topology approach and the Marshall language by Andrej Bauer and Paul Taylor, which takes the approach of semi-decision. This means that one needs not provide proofs of validity of definitions made in the language. Incorrect definitions either return inconsistent/incorrect results or fail to terminate. However, I imagine that the tradeoff is that programs can be slower than if the other approach were taken.
