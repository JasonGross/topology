
There are many approaches to developing topology in a manner that is in some sense ``constructive'', with the names of some principal contributors to each:

\begin{itemize}
\item Locale theory (Johnstone, Vickers)
\item Formal topology (Sambin, Vickers, Maietti, Palmgren)
\hrule
\item Abstract stone duality (Taylor)
\item Synthetic topology (Escard\'o, Bauer) 
\item natural topology (Waaldijk)
\item computable topology / represented spaces (Weihrauch) 
\end{itemize}

I claim that we can classify each theory as following as characterizing ``openness'' either by \emph{serendipity} or by \emph{semi-decision}. Formal topology and locale theory characterize openness by serendipity, while in the rest, opens are semi-decidable properties.

In the serendipitous approaches,
\begin{itemize}
\item Opens are closed under either arbitrary (impredicative) or set-indexed (predicative) joins
\item Points of the Sierp\'inski space are propositions (impredicative) or types (predicative). Two propositions (types) $U$ and $V$ are considered equal as points of the Sierp\'inski space if there are implications (constructions) $U \to V$ and $V \to U$.
\item Spaces without points may have non-trivial structure; two spaces which each have no points are not necessarily homeomorphic.
\end{itemize}

On the other hand, when opens are characterized as semi-deciders,
\begin{itemize}
\item Opens are closed under countable (internal view) or recursively enumerated (external view) joins
\item Points of the Sierp\'inski space are sequences $\nat \to \bool$, where a sequence is the ``true'' point if it has a $\mathsf{true}$ in it, and is false if it is is the constant sequence of all $\mathsf{false}$s.
\item Any two spaces which have no points are homeomorphic.
\end{itemize}

The slogan \emph{serendipity} comes from Steve Vickers. Vickers used the term \emph{serendipity} to indicate this distinction from semi-decidability, saying one of the concepts characterizing ``observability'' is \cite{vickersGeoThDB}
\begin{quote}
\emph{serendipity}: one is told how to know in retrospect when one has observed something, but not any method that's guaranteed to result in the observation whenever possible.
\end{quote}

In another example, Vickers gives as an example of an observable property the existence of the Loch Ness monster, saying that
\begin{quote}
we can well imagine under what conditions we could know that the assertion is true. To be sure, this would take a lot of luck, serendipity or Divine Grace, but no particular effort -- in fact we can imagine others putting much time and resources into a vain attempt to repeat our observation.
\end{quote}

Synthetic topology and abstract stone duality appear to accommodate both views by only requiring that $\Sigma$ forms a \emph{dominance} such that $\nat$ is overt. Depending on the model of synthetic topology, it may be the case that in fact $\Sigma$ has no other truth values than those required to make $\nat$ overt, in which case it is [cite http://math.andrej.com/wp-content/uploads/2010/01/csms_in_synthtop.pdf].

The connection with proofs should be obvious. An observable property is one for which we can check (i.e., decide) whether a putative proof is in fact a valid proof. So any type or proposition in Coq corresponds informally to an observable property in this sense. Now, given any proposition for which proof checking is decidable, we can semi-decide whether the proposition in fact holds, by checking all possible proofs. This is the connection between serendipity and semi-decision. For instance, consider the ``or'' operation. In the ``serendipity'' view, we serendipitously hope that we fall into either of the two explicit cases of the overlapping pattern match (rather than the catch-all case, which of course we fall in to), and if we are so lucky to do so, then our output will also luckily fall into the smaller open of the Sierp\'inski space. In effect, what the $\mathsf{cases}$ definition tells us is when it suffices to have proved the disjunction. The semi-decision view, instead of considering criteria for satisfactory proof, works with semi-decision procedures for finding proofs themselves, and therefore, to find a proof of the disjunction, the two proof finders must be interleaved.

I happen to favor the \emph{serendipity} view, because one can always go from serendipity to semi-decision, but not the other way around, and also because (at least naive) use of the semi-decision approach could result in some very slow programs, whereas in the serendipity point of view, we are required to provide proofs in order for computation to proceed. This is more work (you might complain about writing proofs), but it means that the resulting code will be more efficient. [Explaining this with an example would be a good idea.]

Any point of the ``semi-decision'' Sierp\'inski space can be converted to one in the ``serendipity'' Sierp\'inski space, but not necessarily the other way around. However, from an external perspective, searching for proofs (or inhabitants of types) is semi-decidable, giving a sort of connection in the other direction. 

This distinction is a key difference between the formal topology approach and the Marshall language by Andrej Bauer and Paul Taylor, which takes the approach of semi-decision. This means that one needs not provide proofs of validity of definitions made in the language. Incorrect definitions either return inconsistent/incorrect results or fail to terminate. However, I imagine that the tradeoff is that programs can be slower than if the other approach were taken.
